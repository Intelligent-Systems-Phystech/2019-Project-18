\documentclass[12pt,twoside]{article}
\usepackage{jmlda}
\usepackage[utf8]{inputenc}
\usepackage[russian]{babel}
\usepackage[T2A]{fontenc}
\usepackage{lineno}
\linenumbers
\usepackage{setspace}
\doublespacing

\usepackage[left=1.5cm,right=1.5cm,
    top=2cm,bottom=2cm,bindingoffset=0cm]{geometry}

\title
    [Построение оптимальной системы нейрокомпьютерного интерфейса]
    {Прогнозирование намерений. Построение оптимальной модели декодирования сигналов при моделировании нейрокомпьютерного интерфейса.}
\author {Кудрявцева П.Ю.} % основной список авторов, выводимый в оглавление
\email{polinakud13@gmail.com}
\thanks
    {Научный руководитель:  Стрижов~В.\,В.
    Консультант:  Исаченко~Р.\,В.}
\organization
    {$^1$Московский физико-технический институт (МФТИ)}
\abstract
    {При построении систем нейрокомпьютерного интерфейса возникает проблема наличия зависимости в исходных данных. Для построения устойчивой прогностической модели необходимо снизить размерность исходных данных и провести процедуру выбора признаков. В работе предлагается систематический подход к построению модели, учитывающий зависимости в исходном пространстве сигналов и пространстве целевой переменной. Исследуются алгоритмы построения систем нейрокомпьютерного интерфейса, проведен подробный анализ ошибки различных алгоритмов на реальных данных. Исследовано влияние тензорной структуры данных на качество модели.

\bigskip
\textbf{Ключевые слова}: \emph {отбор признаков, нейрокомпьютерный интерфейс, предсказание движений конечности}.}

\begin{document}
\maketitle



\section{Введение}
 Цель работы - построить систему нейрокомпьютерного интерфейса. Предлагается декодировать сигналы мозга ECoG/EEG и спрогнозировать движение конечности субъекта. Главной проблемой  является наличие сильной корреляции исходных сигналов. Модель, обученная на избыточных данных, является нестабильной. Необходимо избавиться от лишних зависимостей. Для этого применяются методы снижения размерности пространства и выбора признаков.
 
 В качестве алгоритма снижения размерности в работе используется метод частичных наименьших квадратов.Этот метод позволяет получить более эффективные и информативные комбинации старых признаков, вместо того чтобы взять старые признаки без изменений. Алгоритм проецирует признаковую матрицу $X$ и целевую матрицу $Y$ в пространство меньшей размерности, сохраняя максимальное количество информации об исходных матрицах. В новом пространстве признаки в проекции матрицы $X$ линейно независимы. Также, максимизируется взаимосвязь между проекциями. После проецирования максимизируется линейная зависимость между столбцами проекций \cite{pls_effective}\cite{pls_ni}. В \cite{elisey} доказана эффективность рекурсивного PLS для быстрой реакции системы НКИ на поступающие сигналы. В \cite{elisey2}, \cite{elisey3}  предложены многомодальные модификации алгоритма PLS. 
 
 Выбор признаков осуществляется алгоритмом выбора признаков с помощью квадратичного программирования (QPFC). Он позволяет сформулировать задачу отбора признаков в виде задачи квадратичного программирования, которая ставится в пространстве меньшей размерности. Возврат в пространство большей размерности после решения задачи позволяет выбрать независимые признаки, релевантные целевому вектору \cite{qpfc}. В \cite{motrenko} исследуется задача выбора признаков для построения систем нейрокомпьютерных интерфейсов. В статье представлена модификация алгоритма QPFS, которая позволяет применять его к многомерным данным, в том числе для задачи построения НКИ. Доказана эффективность модификации алгоритма для большого количества признаков, по сравнению с другими алгоритмами снижения размерности.
 
 В этой статье объединены существующие подходы к решению задачи построения системы НКИ. Предлагается стандарт для декодирования сигналов ECoG/EEG. Предложен алгоритм анализа имеющихся зависимостей в исходных данных. 

\bibliography{Kudryavtseva2019Project18}{}
\bibliographystyle{plain}
\end{document}
