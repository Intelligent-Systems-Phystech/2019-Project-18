\documentclass[12pt,twoside]{article}
\usepackage{jmlda}
\usepackage[utf8]{inputenc}
\usepackage[russian]{babel}
\usepackage[T2A]{fontenc}
\usepackage{lineno}
\linenumbers
\usepackage{setspace}
\doublespacing


\usepackage[left=1.5cm,right=1.5cm,
    top=2cm,bottom=2cm,bindingoffset=0cm]{geometry}

\newcommand{\norm}[1]{\left\lVert#1\right\rVert}


\title
    [Построение оптимальной системы нейрокомпьютерного интерфейса]
    {Прогнозирование намерений. Построение оптимальной модели декодирования сигналов при моделировании нейрокомпьютерного интерфейса.}
\author {Кудрявцева П.Ю.} % основной список авторов, выводимый в оглавление
\email{polinakud13@gmail.com}
\thanks
    {Научный руководитель:  Стрижов~В.\,В.
    Консультант:  Исаченко~Р.\,В.}
\organization
    {$^1$Московский физико-технический институт (МФТИ)}
\abstract
    {При построении систем нейрокомпьютерного интерфейса возникает проблема наличия зависимости в исходных данных. Для построения устойчивой прогностической модели необходимо провести процедуру выбора признаков и снизить размерность исходных данных. В работе предлагается систематический подход к построению модели, учитывающий зависимости в исходном пространстве сигналов и в пространстве целевой переменной. Исследуются алгоритмы построения систем нейрокомпьютерного интерфейса, проведен подробный анализ ошибки различных алгоритмов на реальных данных. Исследовано влияние тензорной структуры данных на качество модели.

\bigskip
\textbf{Ключевые слова}: \emph {отбор признаков, нейрокомпьютерный интерфейс, PLS, QPFS}.}

\begin{document}
\maketitle



\section{Введение}
 Цель работы ~-- построить систему нейрокомпьютерного интерфейса (НКИ). Предлагается декодировать сигналы мозга ECoG/EEG и спрогнозировать движение конечности субъекта. Главной проблемой  является наличие сильной корреляции исходных сигналов. Модель, обученная на избыточных данных, является нестабильной. Необходимо избавиться от лишних зависимостей. Для этого применяются методы снижения размерности пространства и выбора признаков.
 
 В качестве алгоритма снижения размерности в работе используется метод частичных наименьших квадратов \cite{pls_effective}. Этот метод позволяет получить эффективные и информативные линейные комбинации старых признаков. Алгоритм проецирует признаковую матрицу  \textbf{X}  и целевую матрицу  \textbf{Y}  в пространство меньшей размерности, сохраняя максимальное количество информации об исходных матрицах. В новом пространстве признаки в проекции матрицы \textbf{X} линейно независимы. Также, максимизируется взаимосвязь между проекциями. После проецирования максимизируется линейная зависимость между столбцами проекций \cite{pls_ni}. В \cite{elisey} доказана эффективность рекурсивного PLS для быстрой реакции системы НКИ на поступающие сигналы. В \cite{elisey2}, \cite{elisey3}  предложены многомодальные модификации алгоритма PLS. 
 
 Выбор признаков осуществляется алгоритмом выбора признаков с помощью квадратичного программирования (QPFS). Алгоритм сформулирует задачу отбора признаков в виде задачи квадратичного программирования. Решение этой задачи позволяет выбрать независимые признаки, релевантные целевому вектору \cite{qpfc}. В \cite{motrenko} исследуется задача выбора признаков для построения систем нейрокомпьютерных интерфейсов. В статье представлена модификация алгоритма QPFS, которая позволяет применять его к многомерным данным, в том числе для задачи построения НКИ. Доказана эффективность модификации алгоритма для большого количества признаков, по сравнению с другими алгоритмами снижения размерности.
 
 В этой статье объединены существующие подходы к решению задачи построения системы НКИ. Предлагается стандарт для декодирования сигналов ECoG/EEG. Предложен алгоритм анализа имеющихся зависимостей в исходных данных. 

\section{Постановка задачи}

Данные ECoG состоят из многомерных временных рядов, содержащих информацию о напряжении на 32 элеткродах. Задача заключается в предсказании по сигналам ECoG позиции рук субъекта в следующие моменты времени. По данным строится матрица 
признаков \textbf{X} с m наблюдениями и целевых значений \textbf{Y} = ($y_1,\dots, y_m)$. 
Необходимо учитывать корреляцию данных на этапе построения модели и использовать алгоритмы,
понижающие размерность, для получения новой матрицы признаков \textbf{X} и новой целевой
матрицы \textbf{Y}.  Для этого проводится
отбор признаков с использованием алгоритмов PLS и QPFS. Обозначим новые размерности матриц: $\textbf{Y} \in \mathbb{R}^{m\times r}$, $\textbf{X} \in \mathbb{R}^{m\times n}$. После предобработки исходных данных и выделения оптимального признакового пространства предполагается, что между новой матрицей признаков \textbf{X} и \textbf{Y} существует линейная зависимость: $$ \textbf{Y} = \Theta \textbf{X} + \epsilon $$, где $\epsilon \in \mathbb{R}^{m\times r}$ ~-- вектор остатков, $\Theta \in \mathbb{R}^{r\times n}$ ~-- параметры модели. 

Задача ~-- найти параметр $\Theta$ по данным \textbf{X}, \textbf{Y}. Оптимальные параметры определяются минимизацией квадратичной функции ошибки:
$$ L(\Theta, \textbf{X}, \textbf{Y}) = \norm{\Theta \textbf{X} - \textbf{Y}}^2_2 \to \min_{\Theta} $$


\bibliography{Kudryavtseva2019Project18}{}
\bibliographystyle{plain}
\end{document}
