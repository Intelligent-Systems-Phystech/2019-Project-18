\documentclass[12pt,twoside]{article}

\usepackage[utf8]{inputenc}
\usepackage[russian]{babel}
\usepackage[T2A]{fontenc}
\usepackage{lineno}
\linenumbers
\usepackage{setspace}
\doublespacing

\usepackage[left=1.5cm,right=1.5cm,
    top=2cm,bottom=2cm,bindingoffset=0cm]{geometry}
%\NOREVIEWERNOTES
\title
    {Прогнозирование намерений. Построение оптимальной модели декодирования сигналов при моделировании нейрокомпьютерного интерфейса.}
\author {Кудрявцева П.Ю.} % основной список авторов, выводимый в оглавление


\begin{document}
\maketitle

\abstract
    {При построении систем нейрокомпьютерного интерфейса возникает проблема зависимости считываемых данных. Для построения устойчивой регрессионной модели необходимо уметь снижать размерность исходных данных и отбирать признаки. В работе исследуются алгоритмы построения систем нейрокомпьютерного интерфейса, проведен подробный анализ ошибки различных алгоритмов на реальных данных. Также исследовано влияние тензорной структуры данных на качество модели.

\bigskip
\textbf{Ключевые слова}: \emph {Отбор признаков, Нейрокомпьютерный интерфейс, Предсказание движений конечности}.}

\section{Введение}
 Цель работы - построить нейрокомпьютерный интерфейс, который по сигналам мозга ECoG/EEG должен смоделировать движение конечностью субъекта. Главной проблемой этой задачи является то, что сигналы, считанные с мозга сильно коррелируют между собой, соответственно у нас много избыточных данных. Модель, обученная по избыточным данным, является нестабильной, поэтому необходимо научиться избавляться от лишних данных, например, путем снижения размерности пространства признаков. 
 
 Работа опирается на другие работы в смежных областях, и на статьи, исследующие алгоритмы, используемые далее в работе. В работе будет использован метод частичных наименьших квадратов, который доказал свою эффективность в работе с подобными данными \cite{pls_effective}. Также будет использован алгоритм Quadratic programming feature selection \cite{qpfc}. В статье А.Мотренко и В. Стрижова \cite{motrenko} исследуется похожая задача отбора признаков для построения систем нейрокомпьютерных интерфейсов. 
 
 В этой работе будет использоваться алгоритм частичных наименьших квадратов (PLS) для снижения размерности пространства признаков. Этот метод позволит не только убрать лишние признаки из данных, но и получить более эффективные и информативные комбинации старых признаков, вместо того чтобы взять старые признаки без изменений. Однако, недостатком этого алгоритма является то, что связь новых признаков со старыми неочевидна. Второй алгоритмо который используется в работе - Quadratic Programming Feature Selection (QPFS). Он позволяет сформулировать задачу отбора признаков в виде задачи задачи квадратичного программирования. В статье \cite{qpfc} доказана его эффективность для данных с большими размерами признаков, по сравнению с другими алгоритмами снижения размерности.

\bibliography{Kudryavtseva2019Project18}{}
\bibliographystyle{plain}
\end{document}
